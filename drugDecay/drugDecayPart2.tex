\documentclass[handout,space,nooutcomes]{ximera}
%\documentclass[,nooutcomes]{ximera}

\title{Drug Decay, Part 2}

\begin{document}
\begin{abstract}
Here we investigate various models for the decay of drugs in our bodies.  
\end{abstract}
\maketitle

Pharmacokinetics is the study of the behavior of drugs and other substances administered to living organisms.  
The well-established general model for 
the elimination of a given drug from the body of a given human being is given by the differential equation 
$$y'(t)=-\frac{ky(t)}{A+y(t)}$$
where $y(0)$ is the initial concentration in, say, milligrams per liter, and $y(t)$ is the concentration at time $t$.  The constants $k$ and $A$ depend, of course, on the particular drug and the particular human being. 

%\begin{question}
%Write down some questions about this context.       
%\begin{freeResponse}
%\end{freeResponse}
%\end{question}
%
%
%\begin{question}
%When the drug in question is alcohol, $y(t)$ is usually rather large in comparison to $A$.   Why is it reasonable, in this case, to replace the general model with a simpler model $y'(t)=-k$, where $y(0)$ is the concentration at time $t=0$?  
%\begin{freeResponse}
%\end{freeResponse}
%\end{question}
%
%
%\begin{question}
%What is the general solution to this differential equation $y'(t)=-k$, where $y(0)$ is the concentration at time $t=0$?   Represent your solution symbolically (i.e., as a formula), as a graph, and describe it in words.  
%\begin{freeResponse}
%\end{freeResponse}
%\end{question}
%
%
%\begin{question}
%When the drug in question is cocaine, $y(t)$ is usually very small in comparison to $A$.   Why is it reasonable, in this case, to replace the general model with a simpler model $y'(t)=-\frac{k}{A}y(t)$, where $y(0)$ is the concentration at time $t=0$?  
%\begin{freeResponse}
%\end{freeResponse}
%\end{question}
%
%\begin{question}
%What is the general solution to this differential equation $y'(t)=-\frac{k}{A}y(t)$, where $y(0)$ is the concentration at time $t=0$? 
% Represent your solution symbolically (i.e., as a formula), as a graph, and describe it in words.   
%\begin{freeResponse}
%\end{freeResponse}
%\end{question}

\begin{question}
For most pharmaceuticals, $y(t)$ is usually very small in comparison to $A$.   So again, we replace the general model with a simpler model $y'(t)=-\frac{k}{A}y(t)$, where $y(0)$ is the concentration at time $t=0$.  But many pharmaceuticals are taken not once but at regular intervals over time.  

Suppose a drug is taken once per day, always at the same time.  Draw a graph that shows how drug concentration varies over one week after an initial dose.  Assume that the doses are absorbed immediately into the bloodstream.  
\begin{freeResponse}
\end{freeResponse}
\end{question}

\begin{question}
Write down some additional questions about this context.       
\begin{freeResponse}
\end{freeResponse}
\end{question}

\end{document}
