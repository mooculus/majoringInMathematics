\documentclass[handout,nooutcomes]{ximera}

\graphicspath{{./}{eulerCharacteristic/}{payingOffDebt/}}

%\usepackage{wasysym}
%\newcommand{\pt}{\text{\smiley{}}}
\newcommand{\pt}{\bigstar}


\title[Activity:]{Sums and products combined, oh my!}

\begin{document}
\begin{abstract}
Here we investigate a new way to combine numbers.
\end{abstract}
\maketitle

\paragraph{The setting:}

Suppose you wrote the numbers $1,2,3,\dots,100$ on a chalkboard. You
may take any two numbers, erase them, and then add the sum of their
sum and their product to the list.

\begin{problem}
     Write down as many mathematical questions as you can for this
     setting. After you have your questions, label them as ``Level
     1,'' ``Level 2,'' or ``Level 3'' where:
\begin{description}
\item[Level 1] Means you know the answer, or know exactly how to do
  this problem.
\item[Level 2] Means you think you know how to do the problem.
\item[Level 3] Means you have no idea how to do the problem.
\end{description}
\begin{freeResponse}
  Answers will vary. Try to come-up with $10$ questions of each level!
\end{freeResponse}
\end{problem}


%% \begin{problem}
%% In class we came up with several questions involing this setting. State three of them. 
%% \begin{freeResponse}
%% \end{freeResponse}
%% \end{problem}

%% \begin{problem}
%%   With this setting, we made up a new operation $\pt$ where
%%   \[
%%   a\pt b = a+b + ab
%%   \]
%%   What is
%%   \[
%%   a\pt b\pt c = \answer{a+b+c+ab+ac+bc+abc}
%%   \]
%% \end{problem}

%% \begin{problem}
%% We showed that $\pt$ is \textbf{commutative}, which means
%% \begin{multipleChoice}
%%   \choice[correct]{$a\pt b = b\pt a$}
%%   \choice{$((a\pt b)\pt c) = a(\pt (b\pt c))$}
%% \end{multipleChoice}
%% \end{problem}

%% \begin{problem}
%% We showed that $\pt$ is \textbf{associative}, which means
%% \begin{multipleChoice}
%%   \choice{$a\pt b = b\pt a$}
%%   \choice[correct]{$((a\pt b)\pt c) = a(\pt (b\pt c))$}
%% \end{multipleChoice}
%% \end{problem}

%% \begin{problem}
%% Finally we encoutered several squences of numbers including
%% \begin{align*}
%%   &\{ 1, 3, 7, 15, 63,\dots\}\\
%%   &\{ 2, 4, 8, 16, 64,\dots\}\\
%%   &\{ 1, 5, 23, 119, 719,\dots\}\\
%%   &\{ 1, 2, 6, 24, 120,\dots\}
%% \end{align*}
%% Look these sequences up at \link{https://oeis.org/}{https://oeis.org/}. Find something intersting at the OEIS and tell us about it.
%% \begin{freeResponse}
%% \end{freeResponse}
%% See you next Tuesday!
%% \end{problem}


\end{document}
