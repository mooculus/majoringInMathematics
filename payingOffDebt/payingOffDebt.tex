\documentclass[handout,space,nooutcomes]{ximera}
%\documentclass{ximera}

\title{Paying off Debt}

\begin{document}
\begin{abstract}
Here we investigate where your payments go when you pay off debt.
\end{abstract}
\maketitle

When you take out a loan, the amount of the loan is called
\textbf{principal}.  For most loans, interest is calculated monthly,
based on an \textbf{annual interest rate}, and based on the remaining
balance.  When you make a payment, your payment first offsets the
accumulated interest and then is applied to principal, in order to
calculate the \textbf{remaining principal balance}.  The loan is paid
back in full when the remaining principal balance is reduced to zero.

\begin{question}%[1in]
Suppose you borrow $\$5,000$ and agree to pay it back in equal annual
payments over $5$ years.  Interest is calculated at $7\%$,
compounded annually. 

Write down some questions about this context.  


%%% Without making any calculations, make a guess at what the annual payment should be.  
%%% \begin{freeResponse}
%%% \end{freeResponse}

\end{question}

%
%\begin{question}[0in]
%We will approach this problem numerically.  Use a guess of $\$1250$
%for your annual payment to complete the table below. Round
%all answers to the nearest cent.
%%(Do this work on paper, offline.  In Ximera, enter just the last row of your table.)  
%\[
%  \begin{array}{|c|c|c|c|c|}
%    \hline
%    \text{Year} & \text{Interest}             & \text{Balance with Interest} & \text{Payment}    & \text{Remaining Principal Balance}\\ \hline
%    0  & \$ 0               & ---                   & \$ 0     & \$ 5000 \\ \hline
%    1  & \$ \answer[tolerance=0]{350}    & \$ \answer[tolerance=0]{5350}    & \$ 1250  & \$ \answer[tolerance=0]{4100} \\ \hline
%    2  & \$ \answer[tolerance=0]{287}    & \$ \answer[tolerance=0]{4387}    & \$ 1250  & \$ \answer[tolerance=0]{3137} \\ \hline
%    3  & \$ \answer[tolerance=0]{219.59} & \$ \answer[tolerance=0]{3356.59} & \$ 1250  & \$ \answer[tolerance=0]{2106.59} \\ \hline
%    4  & \$ \answer[tolerance=0]{147.46} & \$ \answer[tolerance=0]{2254.05} & \$ 1250  & \$ \answer[tolerance=0]{1004.05} \\ \hline
%    5  & \$ \answer[tolerance=0]{70.28}  & \$ \answer[tolerance=0]{1074.33} & \$ 1250  & \$ \answer[tolerance=0]{-175.67} \\ \hline
%  \end{array}
%\]
%
%%% \begin{image}
%%% \includegraphics{payingOffDebtTableGraphic.pdf}
%%% \end{image}
%%% \begin{freeResponse}
%%% \end{freeResponse}
%\end{question}
%
%\begin{question}
%Based on your table, was your guess is too low, too high, or just
%right?
%\begin{multipleChoice}
%  \choice{Too low}
%  \choice[correct]{Too high}
%  \choice{Just right}
%\end{multipleChoice}
%
%\end{question}
%
%\begin{question}%[0in]
%  %Complete the table again for other possible payment amounts until
%  %you have found the ``best'' annual payment.  %(In Ximera, enter just
%  %the last row of your table.)
%
%  Complete the table again for a payment
%  of $\$1219.45$.  Round all answers to the nearest cent.
%  \[
%  \begin{array}{|c|c|c|c|c|}
%    \hline
%    \text{Year} & \text{Interest}             & \text{Balance with Interest} & \text{Payment}    & \text{Remaining Principal Balance}\\ \hline
%    0  & \$ 0               & ---                   & \$ 0        & \$ 5000 \\ \hline
%    1  & \$ \answer[tolerance=0]{350}    & \$ \answer[tolerance=0]{5350}    & \$ 1219.45  & \$ \answer[tolerance=0]{4130.55} \\ \hline
%    2  & \$ \answer[tolerance=0]{289.14} & \$ \answer[tolerance=0]{4419.69} & \$ 1219.45  & \$ \answer[tolerance=0]{3200.24} \\ \hline
%    3  & \$ \answer[tolerance=0]{224.02} & \$ \answer[tolerance=0]{3424.26} & \$ 1219.45  & \$ \answer[tolerance=0]{2204.81} \\ \hline
%    4  & \$ \answer[tolerance=0]{154.34} & \$ \answer[tolerance=0]{2359.15} & \$ 1219.45  & \$ \answer[tolerance=0]{1139.70} \\ \hline
%    5  & \$ \answer[tolerance=0]{79.78}  & \$ \answer[tolerance=0]{1219.48} & \$ 1219.45  & \$ \answer[tolerance=0]{0.03} \\ \hline
%    \end{array}
%  \]
%
%%\includegraphics{payingOffDebtTableGraphic.pdf}
%\end{question}
%
%\begin{question}
%Using your best annual payment, how much interest would you pay in total over the $5$ years? 
%\begin{prompt}
%  You would pay \$ $\answer[tolerance=0]{1097.28}$.
%\end{prompt}
%\end{question}
%
%%% \begin{question}[.5in]
%%% Complete the table again, this time with a beginning principal of $P$, an interest rate of $r$,
%%% and a payment amount of $x$.  Hint: It helps to collect factors of $(1+r)$.
%
%%% Then write an equation that must be true if $x$ is the correct
%%% payment amount.   %(In Ximera, enter just the equation.)
%
%%% \includegraphics{payingOffDebtTableGraphic2.pdf}
%
%%% \begin{freeResponse}
%%% \end{freeResponse}
%%% \end{question}
%
%%% \begin{question}
%%% What is your next question?   
%%% \begin{freeResponse}
%%% \end{freeResponse}
%%% \end{question}
%
%
%%\begin{question}
%%Some students notice that the equation above includes a geometric series.  Write the geometric series here.
%%\begin{freeResponse}
%%\end{freeResponse}
%%\end{question}
%%
%%\begin{question}
%%How do you know this is a series?  How do you know this is a geometric series?   
%%\begin{freeResponse}
%%\end{freeResponse}
%%\end{question}
%%
%%\begin{question}
%%Call the geometric series $S$, and note that if you multiply $S$ by $1.07$, the ``common ratio,'' you get another geometric series with many of the same terms.  Use these observations to find the sum of the series.   
%%\begin{freeResponse}
%%\end{freeResponse}
%%\end{question}


\end{document}
